\documentclass[fullpage]{article}
\usepackage{listings}
\usepackage{amsmath}
\usepackage{txfonts}
\input{haskell_style}
\input{yaml_style}
\lstMakeShortInline|
\long\def\ignore#1{}
\title{Experiments with Linear Haskell}
\begin{document}
\maketitle
\tableofcontents
\section{Setting up Stack and Nix for Linear Haskell}
The easiest way to get started with Linear Haskell is using Stack and Nix. We need to use the experimental branch of GHC that supports the \texttt{LinearTypes} extension, and we also need to include the experimental \texttt{linear-base} package. To do this, create a \texttt{stack.yaml} file in your project root matching what is shown below:
\lstinputlisting[language=yaml]{stack.yaml}
You also need to include \texttt{linear-base} in your \texttt{cabal.project} file like so:
\lstinputlisting[language=yaml]{cabal.project}
And finally you need to set up your project's cabal file. For this repository, it is called \texttt{linear-haskell-experiments.cabal} and the contents are as follows:
\lstinputlisting[language=yaml]{linear-haskell-experiments.cabal}
At this point, you can run \texttt{stack build} to build your project with the Linear Haskell branch of GHC, or you can run the repl with \texttt{stack repl} (and enable linear types with \texttt{:set -XLinearTypes}).
\input{experiments/Simple/Pure.lhs}
\input{experiments/Simple/BinaryTree.lhs}
\end{document}
