
\usepackage{upquote}
\usepackage{listings}
% This is used in the acmart.cls, leading to problems with option clash:
\usepackage[usenames,dvipsnames]{xcolor}
% \usepackage[dvipsnames]{xcolor}
% \usepackage[]{xcolor}
% [2017.01.16] Work around option clash:
\definecolor{MidnightBlue}{rgb}{0.1, 0.1, 0.44}

\definecolor{darkgreen}{rgb}{0,0.5,0}
\definecolor{darkred}{rgb}{0.5,0,0}

% Define the language styles we will use
%
\lstset{%
    frame=none,
    rulecolor={\color[gray]{0.7}},
    numbers=none,
    % basicstyle=\footnotesize\ttfamily,
    basicstyle=\mdseries\ttfamily,
%    basicstyle=\scriptsize\ttfamily,
    numberstyle=\color{Gray}\it,
    commentstyle=\color{MidnightBlue}\it,
    stringstyle=\color{Maroon},
    keywordstyle=[1],
    keywordstyle=[2]\bfseries\color{ForestGreen},
    keywordstyle=[3]\bfseries\color{Bittersweet},
    keywordstyle=[4]\bfseries\color{RoyalPurple},
    captionpos=b,
    aboveskip=1\medskipamount,
    xleftmargin=0.5\parindent,
    xrightmargin=0.5\parindent,
    flexiblecolumns=false,
%   basewidth={0.5em,0.45em},           % default {0.6,0.45}
%    escapechar={\%},
    % escapechar={\@},
    % mathescape=true,
    texcl=true,                          % tex comment lines
    showstringspaces=false
}

\lstloadlanguages{Haskell}
\lstdefinestyle{haskell}{%
    language=Haskell,
    upquote=true,
    deletekeywords={case,class,data,default,deriving,do,in,instance,let,of,type,where,IO,ST,STM},
    morekeywords={[2]class,data,default,deriving,family,instance,type,where},
    morekeywords={[3]in,let,case,of,do,letpacked,letloc,letregion},
    morekeywords={[4]IORef,IO,ST,STM,STRef,TVar},
    literate=
        {\\}{{$\lambda$}}1
        {Lambda}{{$\Lambda$}}1
        {\\\\}{{\char`\\\char`\\}}1
        {>->}{>->}3
        {>>=}{>>=}3
        {->}{{$\rightarrow$}}2
        {>=}{{$\geq$}}2
        {<-}{{$\leftarrow$}}2
        {<=}{{$\leq$}}2
        {=>}{{$\Rightarrow$}}2
        {|}{{$\mid$}}1
        {forall}{{$\forall$}}1
        {exists}{{$\exists$}}1
        {...}{{$\cdots$}}3
        {\#->}{{$\multimap\,\,$}}3
%       {`}{{\`{}}}1
%       {\ .}{{$\circ$}}2
%       {\ .\ }{{$\circ$}}2
%
%    deletekeywords={insert},
%    deletekeywords={map,sort,zipWith,replicate,Num,Char,Bool,Array,Int,Double
%                   ,sqrt,not,filter,IO,Maybe,Either,quot,scanl,scanr,reverse,fst,id},
%    literate=
%        {+}{{$+$}}1
%        {/}{{$/$}}1
%        {*}{{$*$}}1
%        % {=}{{$=$}}1
%        {>}{{$>$}}1 {<}{{$<$}}1
%        {\\}{{$\lambda$}}1
%        {\\\\}{{\char`\\\char`\\}}1
%        {->}{{$\rightarrow\;$}}2
%        {>=}{{$\geq$}}2
%        {<-}{{$\leftarrow\;$}}2
%        {<=}{{$\leq$}}2
%        {=>}{{$\Rightarrow\;$}}2
%        {\ .}{{$\circ$}}2
%        {\ .\ }{{$\circ$}}2
%        {>>}{{>>}}2
%        {>>=}{{>>=}}2
%        {=<<}{{=<<}}2
%        {|}{{$\mid$}}1
%        {dotdotdot}{{$\ldots$}}3
}

\lstdefinestyle{inline}{%
    style=haskell,
    basicstyle=\mdseries\ttfamily,
    %% keywordstyle=[1],
    %% keywordstyle=[2],
    %% keywordstyle=[3],
    %% keywordstyle=[4],
    % escapechar={\@},
    % mathescape=true,
    literate=
        {\\}{{$\lambda$}}1
        {Lambda}{{$\Lambda$}}1        
        {\\\\}{{\char`\\\char`\\}}1
        {>->}{>->}3
        {>>=}{>>=}3
        {->}{{$\rightarrow$\space}}3    % include forced space
        {>=}{{$\geq$}}2
        {<-}{{$\leftarrow$}}2
        {<=}{{$\leq$}}2
        {=>}{{$\Rightarrow$}}2
        {|}{{$\mid$}}1
%        {~}{{$\sim$}}1
        {forall}{{$\forall$}}1
        {exists}{{$\exists$}}1
        {...}{{$\cdots$}}3
}

\lstnewenvironment{code}
    {\lstset{style=haskell}%
      \csname lst@SetFirstLabel\endcsname}
    {\csname lst@SaveFirstLabel\endcsname}
    {}
\lstnewenvironment{spec}
    {\lstset{style=haskell}%
      \csname lst@SetFirstLabel\endcsname}
    {\csname lst@SaveFirstLabel\endcsname}
    {}

% Default all listings to Haskell style
\lstset{style=haskell}

% \newcommand{\inl}[1]{\lstinline[style=inline];#1;}
\newcommand{\il}[1]{\lstinline[style=inline,mathescape=true];#1;}

\newcommand{\makeatcode}{\lstMakeShortInline[style=inline]@}
\newcommand{\makeatchar}{\lstDeleteShortInline@}
